\subsection{Misura Clockwise}
%Gioacchino Rossini
La misura della velocità della luce si effettua misurando lo spostamento ,dovuto alla rotazione dello speccio, di un fascio di luce laser.
Settando sul pannello del motore sia la direzione in senso orario (CW) che una frequenza nell'intervallo dei $70\sim 110\hertz$, per mezzo del micrometro fissato all'oculare, si centra il mirino nel centro del fascio laser.\\
Dopo aver preso la misura della posizione, si aumenta la frequenza di rotazione al valore massimo di circa 1070\hertz, utilizzando nuovamente il mirino dell'oculare, si centra nuovamente la macchia che si sarà spostata in alto o in basso rispetto alla posizione iniziale $x_0$ di un valore misurabile $\delta x$ ottenibile leggendo lo spostamento sul micrometro.\\
A questo punto è possibile premere il pulsante di regime limite del motore, regime che non può essere mantenuto a lungo per evitare surriscaldamento e conseguente danno al motore, e per mezzo dell'oculare sarà possibile vedere uno spostamento $\delta x_1$ ulteriore della macchia, sempre misurabile per mezzo del micrometro.
