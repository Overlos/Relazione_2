\subsection{Descrizoone dell'apparato}
I principali strumenti di cui consiste l'apparato sperimentale sono:
\begin{itemize}
\item Microscopio con crucifilo;
\item Splitter;
\item 2 specchi piani;
\item specchio concavo;
\item specchio rotante;
\item motore ;
\item laser;
\item 2  Lenti convesse ;
\item base con binario magnetico;
\item Metro;
\item 2 supporti per gli specchi;
\item quel coso che impone la rotazione allo specchio %??????????
\end{itemize}
La loro disposizione è mostrata in %ref figura.
%% inizio didascalia immagine %%
Sulla base sono ancorati il laser, il microscopio con lo splitter e lo specchio rotante col proprio motore; su un lato della base è fissato il metro, che consente 
di posizionare con precisione le lenti. Queste vengono fissate sulla base grazie al binario magnetico. Gli specchi piani e lo specchio concavo sono fissati sui 
rispettivi supporti. Delle viti disposte dietro gli specchi consentono di modificarne l'inclinazione. Microscopio e splitter sono  inseriti su un unico supporto
dotato di vite micrometrica che consente di spostare il microscopio.
%% fine didascalia immagine %%


In aggiunta, per la calibrazione dell'apparato sono richiesti:
\begin{itemize}
\item Metro 
\item due lenti polaroid 
\item squadretta 
\end{itemize}
Il metro è utile misurare la distanza complessiva percorsa dalla luce; le lenti polaroid servono a ridurre l'intensità del raggio laser per poterlo osservare senza
rischio; la squadretta invece consente di verificare che il raggio laser sia centrato sullo specchio rotante con la precisione del millimetro.
